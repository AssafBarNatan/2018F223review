\documentclass[red]{tutorial}
\usepackage[no-math]{fontspec}
\usepackage{xpatch}
\renewcommand{\ttdefault}{ul9}
\xpatchcmd{\ttfamily}{\selectfont}{\fontencoding{T1}\selectfont}{}{}
\DeclareTextCommand{\nobreakspace}{T1}{\leavevmode\nobreak\ }
\usepackage{polyglossia} % English please
\setdefaultlanguage[variant=us]{english}
%\usepackage[charter,cal=cmcal]{mathdesign} %different font
%\usepackage{avant}
\usepackage{microtype} % Fewer badboxes


\usepackage[charter,cal=cmcal]{mathdesign} %different font
%\usepackage{euler}

\usepackage{blindtext}
\usepackage{calc, ifthen, xparse, xspace}
\usepackage{makeidx}
\usepackage[hidelinks, urlcolor=blue]{hyperref}   % Internal hyperlinks
\usepackage{mathtools} % replaces amsmath
\usepackage{bbm} %lower case blackboard font
\usepackage{amsthm, bm}
\usepackage{thmtools} % be able to repeat a theorem
\usepackage{thm-restate}
\usepackage{graphicx}
\usepackage{xcolor}
\usepackage{multicol}
\usepackage{fnpct} % fancy footnote spacing
\usepackage{textcomp} %turn copyright into copyleft


\newcommand{\xh}{{{\mathbf e}_1}}
\newcommand{\yh}{{{\mathbf e}_2}}
\newcommand{\zh}{{{\mathbf e}_3}}
\newcommand{\R}{\mathbb{R}}
\newcommand{\Z}{\mathbb{Z}}
\newcommand{\N}{\mathbb{N}}
\newcommand{\mat}[1]{\begin{bmatrix} #1 %
\end{bmatrix}}
\newcommand{\mute}[1]{}

\DeclareMathOperator{\proj} {proj}
\DeclareMathOperator{\Proj} {proj}
\DeclareMathOperator{\Perp} {perp}
\DeclareMathOperator{\Span} {span}
\DeclareMathOperator{\Img}  {img}
\DeclareMathOperator{\Null} {null}
\DeclareMathOperator{\Range}{range}
\DeclareMathOperator{\rref} {rref}
\DeclareMathOperator{\rank} {rank}
\DeclareMathOperator{\Rank} {rank}
\DeclareMathOperator{\nnul} {nullity}
\DeclareMathOperator{\chr}  {char}

\renewcommand{\d}{\mathrm{d}}


\theoremstyle{definition}
\newtheorem{example}{Example}[section]
\newtheorem{defn}{Definition}[section]

\theoremstyle{theorem}
\newtheorem{thm}{Theorem}[section]

\pgfkeys{/tutorial,
name={Final Review Session},
author={The MAT223 Team},
course={MAT 223},
date={December 6-7},
term={Fall 2018},
title={Linear somegebra? Linear \emph{all}gebra.}
}

\begin{document}
\begin{tutorial}
  \mute{
  %Is writing all the learning objectives necessary?
  \begin{objectives} Is writing all the learning objectives necessary?
    In this session you will be working with linear algebra.
    Recall the course learning objectives:
    \begin{itemize}
      \item
        \emph{Work independently to understand concepts
        and procedures that have not been previously explained to you.}
      \item
        \emph{Clearly and correctly express the mathematical ideas of linear
        algebra to others, and understand and apply logical arguments and
        definitions that have been written by others.}
      \item
        \emph{Translate between algebraic and geometric viewpoints to solve
        problems.}
      \item
        \emph{Use matrices, matrix arithmetic, matrix
        inverses, systems of linear equations, row reduction,
        determinants, and eigenvalues and eigenvectors
        to solve problems. As well, write vectors in different
        bases and pick an appropriate basis when working on problems.}
    \end{itemize}
  \end{objectives}}
  \begin{objectives}
    This review session has one objective, and that is to 
    \emph{Work in groups to identify course objectives that you 
    are not comfortable with}.
  \end{objectives}


  \begin{enumerate}
    \item The following sentences have errors or are missing 
      something. Correct them or add the missing details.
      \begin{enumerate}
          \mute{
        \item \textbf{(Linear Transformations)}\\
          %% Do we need this? Does this do the same as almost 
          %% linear question?
          % tests isolating ambiguity and finding missing quantification
          A transformation $f\colon\R^n\to \R^m$ is linear if 
          it sends sums to sums, $f(\vec0) = \vec0$, and 
          $f(\alpha \vec x) = \alpha f(\vec x)$.
        \item 
          % tests broad thinking to remember that a basis is necessary
          % tests mathematical convention
          An $n\times m$ matrix corresponds to a linear 
          transformation from $\R^n$
          to $\R^m$.}
        \item \textbf{(Subspaces \& Bases)}\\
          % tests knowing the definition, and ignoring irrelevant info
          A basis for the subspace $V$ is a set $S$ 
          such that $\dim S = \dim V$ and $\Span(S) = V$.
        \item \textbf{(Vectors)}\\
          % tests type analysis and wrong quantifier
          $\vec w$ is a convex combination of $u$ and $v$ if:
          \begin{equation*}
            \vec w =
            \bigl\{\,\vec w:\vec w = a\vec u %
              +b\vec v \text{ for all } a,b\in [0,1]\,
            \bigr\}
          \end{equation*}
        \item \textbf{(Determinants; Dot products)} \\
          %orthogonality, algebra, and geometry
          If $\vec a$, $\vec b$, and $\vec c$ are pairwise orthogonal 
          vectors in $\R^3$, then 
          $\det \left[\vec a\;\middle|\;\vec b\;\middle|\;\vec c\right] = 1$.
      \end{enumerate}
      \mute{
    \item % Not a review question.
      Sam has a function $T\colon\R^n\to \R^m$, which has the following
      property:
      \begin{center}
        For any subspace $V\subset \R^n$, we have $T(V)$ is a subspace of $\R^m$.
      \end{center}
      \begin{enumerate}
          % use the "definition of linearity"
          % understand the difference between sets and their contents
        \item What are the possible values of $T(\vec0)$?
        \item Sam says that their function is not linear. Can this be true?
      \end{enumerate}
      }
    \item \textbf{(Linear Transformations)}\\
      % use definition of linearity, and know how to check 
      % basic conditions
      A transformation $T:\R^n\to \R^m$ is called 
      \textit{almost linear} if for all $x,y\in \R^n$, we have 
      $\|T(x+y)\| \le \|T(x)+T(y)\|$, and for all 
      $\alpha\ge 0$ and $x\in \R^n$, we have 
      $\|T(\alpha x)\|\le \alpha \|T(x)\|$.
      \begin{itemize}
        \item Write down a clear and concise explanation of what 
          one must do in order to show that a transformation is 
          almost linear.
        \item Show that the transformation $f:\R\to\R$ defined 
          by $f(x) = x^2$ is not almost linear.
        \item Show that any linear transformation in almost linear.
      \end{itemize}
    \item \textbf{(Inverses)}\\
      \newcommand{\pmin}{\phantom{-}}
      \newcommand{\row}[1]{\mathrm{r}_{#1}}
      Mohammed has partially row-reduced the matrix $A$ using the
      following steps. Use his work to write $A$ as a product of elementary
      matrices. Use this to compute $\det A$.
      \begin{align*}
        A = \begin{bmatrix}
          0 & 0 & 0 & 2 \\
          0 & 1 & 0 & 0 \\
          1 & 0 & 0 & 0 \\
          1 & 0 & 3 & 0
        \end{bmatrix}&\xrightarrow{\row{1} \leftrightarrow \row{3}}
        \begin{bmatrix}
          1 & 0 & 0 & 0 \\
          0 & 1 & 0 & 0 \\
          0 & 0 & 0 & 2 \\
          1 & 0 & 3 & 0
        \end{bmatrix}
        \xrightarrow{\row{4} \to \row{4}-\row{1}}
        \begin{bmatrix}
          1 & 0 & 0 & 0 \\
          0 & 1 & 0 & 0 \\
          0 & 0 & 0 & 2 \\
          0 & 0 & 3 & 0
        \end{bmatrix}
        &\xrightarrow{\row{4} \leftrightarrow \row{3}}
        \begin{bmatrix}
          1 & 0 & 0 & 0 \\
          0 & 1 & 0 & 0 \\
          0 & 0 & 3 & 0 \\
          0 & 0 & 0 & 2
        \end{bmatrix}&&
      \end{align*}

    \item \textbf{(Determinants)}\\
      % Jason said he didn't like the second part of this question?
      For which values of $a$ is the matrix $A$ invertible? 
      What is the rank of $A$ when it is not invertible?
      \begin{equation*}
        A = 
        \begin{bmatrix}
          0 & 3 & 2 \\
          a & 0 & 0 \\
          8 & 2 & 0 \\
        \end{bmatrix}
      \end{equation*}
    \item \textbf{(Eigenvalues \& Diagonalization; Similar Matrices)}\\
      Let $T\colon\R^2\to\R^2$ be the transformation which projects vectors
      onto the subspace $V=\Span\{\vec v\}$, where $\vec v = \mat{1\\1}$.
      \begin{enumerate}
        \item Find an appropriate basis to express $T$ as a matrix.
        \item
          What are the eigenvectors of $T$? Is $T$ diagonalizable?
      \end{enumerate}
    \item \textbf{(Computational Objectives, Representations of Lines)}\\
      Let $\vec a = \mat{-1\\0\\1}$ and $\vec b = \mat{1\\0\\0}$. 
      Define:
      \begin{align*}
        V = \{\vec x\in \R^2| \vec x \perp \vec a \text{ and } 
        \vec x \perp \vec b\}
      \end{align*}
      \begin{itemize}
        \item Express $V$ as the solution to an augmented matrix
        \item Express $V$ as a linear system
        \item Express $V$ in parametric form
        \item Express $V$ as the span of vectors
        \item Sketch $V$.
      \end{itemize}
  \end{enumerate}
\end{tutorial}

\mute{%XXX XXX XXX XXX BEGIN MUTE XXX XXX XXX XXX XXX
\begin{solutions}
  Solutions go here.
\end{solutions}
\begin{instructions}
  \subsection*{Learning Objectives}
  Students need to be able to\ldots
  \begin{itemize}
    \item Define eigenvector and eigenvalue
    \item Have an intuitive geometric understanding of eigenvectors
    \item Use the definition of eigenvector to classify vectors as eigenvectors or not
    \item Apply eigenvectors to solve problems
  \end{itemize}

  \subsection*{Context}
  Students have been introduced to determinants, characteristic polynomials, and the definition
  of eigenvectors/values in lecture. Some students may have started diagonalization in
  lecture.

  In this course we will only pursue real eigenvalues/vectors. Some students may know complex
  numbers from high school, but most will not.


  \subsection*{What to Do}
  Start the tutorial by stating the day's learning objectives. Emphasize that eigenvectors/values
  are the capstone of this course and they take a while to get used to. Today we are getting used
  to them.

  Have students start in groups on \#1. They should be used to the style of part (a) by now,
  but they will probably struggle with part (b). Try to get them to produce something better than
  reading the mathematical definition aloud.

  \#2 is a computational question that follows straight from the definition. If you like,
  divide the class into three teams.
  One team will check for $A$, another for $B$, and the last for $C$. Write up (or have each
  team write up) a summary of their results. Then have everyone verify each team's result.

  Before finishing \#2, make sure to have a discussion about $\vec v_6$. Is it an eigenvector?
  Does it go to a multiple of itself?

  It is likely that \#1 and \#2 will take most of class time. If you do have extra time, continue as usual,
  letting students work in small groups on \#3 and then having a mini-discussion when half the groups
  have figured it out.

  6 minutes before the end of class, pick a suitable problem to do as a wrap-up.


  \subsection*{Notes}
  \begin{itemize}
    \item For \#1(b), many students will write ``$\vec v$ is an eigenvector for $T$ if
      $\vec v$ is non-zero and $T$ of $\vec v$ is $\lambda$ times $\vec v$ for some scalar
      $\lambda$.'' This is reading the mathematical definition aloud and is not what we want.
      We want students to rephrase $T\vec v=\lambda \vec v$ in terms of a vector not changing
      direction.
    \item For \#2, some students will get confused between $\vec v\neq \vec 0$ and $\lambda \neq 0$.
      $C,\vec v_4$ has eigenvalue $0$ precisely so this confusion will come up. Make sure to
      have a conversation about this whether or not it comes up naturally.
    \item Encourage students to draw pictures for \#3. They will naturally try to start by writing
      down a matrix for $\mathcal P$ and $\mathcal R$, but that short-circuits thinking.
    \item
      When talking about eigenvalues of a transformation $T$
      not existing, always make sure to say ``$T$ has no real eigenvalues''. Students who
      know about complex numbers can have a deeper conversation with you about them after class.
    \item In \#4, some might find it so ``obvious'' that they don't know how to prove it. That's largely
      the point. They should be able to prove obvious things!
    \item \#5 culminates in diagonalization, but it asks students to make a judgement about whether
      a vector is ``approximated''. The fact that $\|\vec v_1\|,\|\vec v_2\|\leq 100$ is essential
      for this, but isn't the point of the problem. The point is that there is a unique, largest, positive
      eigenvalue.
  \end{itemize}
\end{instructions}
}
%XXX XXX XXX XXX END MUTE XXX XXX XXX XXX XXX}
\end{document}
