\documentclass[red]{tutorial}
\usepackage[no-math]{fontspec}
\usepackage{xpatch}
\renewcommand{\ttdefault}{ul9}
\xpatchcmd{\ttfamily}{\selectfont}{\fontencoding{T1}\selectfont}{}{}
\DeclareTextCommand{\nobreakspace}{T1}{\leavevmode\nobreak\ }
\usepackage{polyglossia} % English please
\setdefaultlanguage[variant=us]{english}
%\usepackage[charter,cal=cmcal]{mathdesign} %different font
%\usepackage{avant}
\usepackage{microtype} % Fewer badboxes


\usepackage[charter,cal=cmcal]{mathdesign} %different font
%\usepackage{euler}

\usepackage{blindtext}
\usepackage{calc, ifthen, xparse, xspace}
\usepackage{makeidx}
\usepackage[hidelinks, urlcolor=blue]{hyperref}   % Internal hyperlinks
\usepackage{mathtools} % replaces amsmath
\usepackage{bbm} %lower case blackboard font
\usepackage{amsthm, bm}
\usepackage{thmtools} % be able to repeat a theorem
\usepackage{thm-restate}
\usepackage{graphicx}
\usepackage{xcolor}
\usepackage{multicol}
\usepackage{fnpct} % fancy footnote spacing
\usepackage{textcomp} %turn copyright into copyleft

\delimitershortfall-1sp % Allow automatic growth of delimiters


\newcommand{\xh}{{{\mathbf e}_1}}
\newcommand{\yh}{{{\mathbf e}_2}}
\newcommand{\zh}{{{\mathbf e}_3}}
\newcommand{\R}{\mathbb{R}}
\newcommand{\Z}{\mathbb{Z}}
\newcommand{\N}{\mathbb{N}}
\newcommand{\mat}[1]{\begin{bmatrix} #1 %
\end{bmatrix}}
\newcommand{\pmin}{\phantom{-}}
\newcommand{\row}[1]{\mathrm{r}_{#1}}
\newcommand{\mute}[1]{}

\DeclarePairedDelimiter\norm{\lVert}{\rVert}
\makeatletter
\let\oldnorm\norm
\def\norm{\@ifstar{\oldnorm}{\oldnorm*}}
\makeatother

\DeclareMathOperator{\proj} {proj}
\DeclareMathOperator{\Proj} {proj}
\DeclareMathOperator{\Perp} {perp}
\DeclareMathOperator{\Span} {span}
\DeclareMathOperator{\Img}  {img}
\DeclareMathOperator{\Null} {null}
\DeclareMathOperator{\Range}{range}
\DeclareMathOperator{\rref} {rref}
\DeclareMathOperator{\rank} {rank}
\DeclareMathOperator{\Rank} {rank}
\DeclareMathOperator{\nnul} {nullity}
\DeclareMathOperator{\chr}  {char}

\renewcommand{\d}{\mathrm{d}}


\theoremstyle{definition}
\newtheorem{example}{Example}[section]
\newtheorem{defn}{Definition}[section]

\theoremstyle{theorem}
\newtheorem{thm}{Theorem}[section]

\pgfkeys{/tutorial,
name={Final Review Session},
author={The MAT223 Team},
course={MAT 223},
date={December 6-7},
term={Fall 2018},
title={Linear somegebra? Linear \emph{all}gebra.}
}

\begin{document}
\begin{tutorial}
  \begin{objectives}
    This review session has one objective, and that is to
    \emph{work in groups to identify learning objectives that you
    are not comfortable with}.
  \end{objectives}
  \begin{enumerate}
    \item\label{q:mistake}
      The following sentences have errors or are missing something. Correct
      them or add the missing details.
      \begin{enumerate}
          \mute{
        \item \textbf{(Linear Transformations)}\\
          %% Do we need this? Does this do the same as almost
          %% linear question?
          % tests isolating ambiguity and finding missing quantification
          A transformation $f\colon\R^n\to \R^m$ is linear if
          it sends sums to sums, $f(\vec0) = \vec0$, and
          $f(\alpha \vec x) = \alpha f(\vec x)$.
        \item
          % tests broad thinking to remember that a basis is necessary
          % tests mathematical convention
          An $n\times m$ matrix corresponds to a linear
          transformation from $\R^n$
          to $\R^m$.
        }
        \item \textbf{(Subspaces \& Bases; Span; Linear Independence)}\\
          % tests knowing the definition, and ignoring irrelevant info
          A basis for the subspace $V$ is a set $S$
          such that $\dim S = \dim V$ and $\Span(S) = V$.
        \item \textbf{(Vectors)}\\
          % tests type analysis and wrong quantifier
          Vector $\vec w$ is a convex combination of $\vec u$ and $\vec v$ if:
          \begin{equation*}
            \vec w =
            \bigl\{\,\vec w:\vec w = a\vec u %
            +b\vec v \text{ for all } a,b\in [0,1]\,
            \bigr\}
          \end{equation*}
        \item \textbf{(Determinants; Dot products)} \\
          %orthogonality, algebra, and geometry
          If $\vec a$, $\vec b$, and $\vec c$ are pairwise orthogonal
          vectors in $\R^3$, then
          $\det \bigl[\vec a\bigm|\vec b\bigm|\vec c\bigr] = 1$.
        \item \textbf{(Eigenvectors \& Diagonalization)}\\
          Every matrix has a basis of eigenvalues.
      \end{enumerate}
      \mute{
    \item % Not a review question.
      Sam has a function $T\colon\R^n\to \R^m$, which has the following
      property:
      \begin{center}
        For any subspace $V\subseteq \R^n$, we have $T(V)$ is a subspace of
        $\R^m$.
      \end{center}
      \begin{enumerate}
          % use the "definition of linearity"
          % understand the difference between sets and their contents
        \item What are the possible values of $T(\vec0)$?
        \item Sam says that their function is not linear. Can this be true?
      \end{enumerate}
      }
    \item \label{q:lin_transform}\textbf{(Linear Transformations)}\\
      % use definition of linearity, and know how to check
      % basic conditions
      A transformation $T\colon\R^n\to \R^m$ is called
      \emph{almost linear} if for all $x$, $y\in \R^n$, and
      $\alpha\ge 0$, we have
      \begin{align*}
        \norm*{T(x+y)} &\le \norm*{T(x)+T(y)}\\
        \norm*{T(\alpha x)}&\le \alpha \norm*{T(x)}
      \end{align*}
      \begin{enumerate}
        \item \label{q:definition_expansion}
          Write down a clear and concise explanation of what one must do in
          order to show that a transformation is almost linear.
        \item Show that the transformation $f\colon\R\to\R$ defined
          by $f(x) = x^2$ is not almost linear.
        \item \label{q:lin_almost_lin}
          Show that any linear transformation is almost linear.
      \end{enumerate}
    \item \label{q:inverses} \textbf{(Inverses)}\\
      Mohammed has partially row-reduced the matrix $A$ using the
      following steps. Use his work to write $A$ as a product of elementary
      matrices. Use this to compute $\det A$.
      \begin{align*}
        A = \begin{bmatrix}
          0 & 0 & 0 & 2 \\
          0 & 1 & 0 & 0 \\
          1 & 0 & 0 & 0 \\
          1 & 0 & 3 & 0
        \end{bmatrix}
        &\xrightarrow{\row{1} \leftrightarrow \row{3}}
        \begin{bmatrix}
          1 & 0 & 0 & 0 \\
          0 & 1 & 0 & 0 \\
          0 & 0 & 0 & 2 \\
          1 & 0 & 3 & 0
        \end{bmatrix}
        \xrightarrow{\row{4} \to \row{4}-\row{1}}
        \begin{bmatrix}
          1 & 0 & 0 & 0 \\
          0 & 1 & 0 & 0 \\
          0 & 0 & 0 & 2 \\
          0 & 0 & 3 & 0
        \end{bmatrix}
        &\xrightarrow{\row{4} \leftrightarrow \row{3}}
        \begin{bmatrix}
          1 & 0 & 0 & 0 \\
          0 & 1 & 0 & 0 \\
          0 & 0 & 3 & 0 \\
          0 & 0 & 0 & 2
        \end{bmatrix}
      \end{align*}
    \item \label{q:determinants} \textbf{(Determinants and Inverses; Subspaces)}\\
      We define:
      \begin{equation*}
        A =
        \mat{
          0 & 3 & 2 \\
          a & 0 & 0 \\
          8 & 2 & 0 \\
        }
      \end{equation*}
      \begin{enumerate}
        \item For which values of $a$ is the matrix $A$ invertible?
        \item What is the rank of $A$ when it is not invertible?
        \item When $A$ is not invertible, find a basis for the range of the
          transformation $T_A(x) = Ax$.
      \end{enumerate}
    \item \textbf{(Eigenvalues \& Diagonalization; Similar Matrices;
      Projections)}\\
      Let $T\colon\R^2\to\R^2$ be the transformation which
      projects vectors onto the subspace $V=\Span\{\vec v\}$,
      where $\vec v = \mat{1\\1}$.
      \begin{enumerate}
        \item \label{q:standard_basis}
          Write $T$ in the standard basis, and in one other basis. Which one do
          you prefer?
        \item Is $T$ one-to-one? Is $T$ onto?
        \item What are the eigenvectors of $T$? Is $T$ diagonalizable?
        \item Is $T$ invertible?
      \end{enumerate}
    \item \textbf{(Computational Objectives; Representations of Lines;
      SLE)}\\
      Let $\vec a = \mat{-1\\0\\1}$ and $\vec b = \mat{1\\0\\0}$.
      Define the set $V$ as
      \begin{align*}
        V = \left\{\,\vec x\in \R^2: %
        (\vec x - \vec e_3)\perp \vec a \text{ and }
        (\vec x - \vec e_3)\perp \vec b \,\right\}
      \end{align*}
      \begin{enumerate}
        \item Write down a system of linear equations where
              $V$ is the complete solution set.
        \item Write down the corresponding augmented matrix.
        \item Express $V$ in vector form.
        \item Express $V$ as the span or translated span of
              vectors.
        \item Sketch $V$.
      \end{enumerate}
  \end{enumerate}
\end{tutorial}

\begin{solutions}
  \begin{enumerate}
    \item Here are the things students should identify for 
      each question:
      \begin{enumerate}
        \item Dimension does not make sense in this context, and 
          the proper definition includes linear independence
        \item $=$ should be $\in$, and the `for all` makes no 
          sense there. Additionally, the condition is wrong. Students 
          should also be able to make this answer shorter.
        \item Students should replace $1$ with $\pm \|a\|\|b\|\|c\|$, 
          and do this by understanding that the unit cube is sent to 
          another cube.
        \item The word `eigenvalues` should be replaced with 
          `eigenvectors,` and even so, it does not make sense 
          for a matrix to have a basis. This question is quite 
          open-ended, and there are many directions the students 
          can take this.
      \end{enumerate}
    \item 
      \begin{enumerate}
        \item To show that a transformation $T$ is almost 
          linear, one needs to show that for any $x,y\in \R^n$ and 
          $\lambda \in \R$ such that $\lambda\ge 0$:
          \begin{itemize}
            \item $\|T(x+y)\| \le \|T(x)+T(y)\|$
            \item $\|T(\lambda x)\| \le \lambda \|T(x)\|$
          \end{itemize}
        \item Take $x = 2$ and $\lambda = 2$. Then 
          $\|f(\lambda x)\| = 16$, but 
          $\lambda \|T(x)\| = 2 \|4\| = 8$.
        \item In a linear transformation, the first condition 
          holds and is an equality. Now, let $\lambda \ge0$ be 
          given, and let $x\in \R^n$. Then by linearity:
          \begin{align*}
            \|T(\lambda x)\| = \|\lambda T(x)\|
            =|\lambda|\|T(x)\| = \lambda \|T(x)\|
          \end{align*}
          Showing almost linearity.
      \end{enumerate}
    \item
      \begin{align*}
        A = 
        \begin{bmatrix}
          0 & 0 & 1 & 0 \\
          0 & 1 & 0 & 0 \\
          1 & 0 & 0 & 0 \\
          0 & 0 & 0 & 1
        \end{bmatrix}
        \begin{bmatrix}
          1 & 0 & 0 & 0 \\
          0 & 1 & 0 & 0 \\
          0 & 0 & 1 & 0 \\
          1 & 0 & 0 & 1
        \end{bmatrix}
        \begin{bmatrix}
          1 & 0 & 0 & 0 \\
          0 & 1 & 0 & 0 \\
          0 & 0 & 0 & 1 \\
          0 & 0 & 1 & 0
        \end{bmatrix}
        \begin{bmatrix}
          1 & 0 & 0 & 0 \\
          0 & 1 & 0 & 0 \\
          0 & 0 & 3 & 0 \\
          0 & 0 & 0 & 1
        \end{bmatrix}
        \begin{bmatrix}
          1 & 0 & 0 & 0 \\
          0 & 1 & 0 & 0 \\
          0 & 0 & 1 & 0 \\
          0 & 0 & 0 & 2
        \end{bmatrix}
      \end{align*}
      and the determinant of $A$ is $6$.
    \item 
      \begin{enumerate}
        \item When $a=0$, the columns of $A$ are linearly dependent, so 
          $A$ is not invertible. When $a\neq 0$, they are linearly 
          independent, so the determinant is nonzero.
        \item When $A$ is not invertible, $a=0$, in which case 
          the columns of $A$ span a $2$-dimensional subspace of 
          $\R^3$, meaning that the rank of $A$ is $2$.
        \item $\left\{\mat{0\\0\\1} , \mat{1\\0\\0}\right\}$
      \end{enumerate}
    \item
      \begin{enumerate}
        \item $[T]_{\mathcal{E}} = \frac12\mat{1&1\\1&1}$, and 
          if $\mathcal{B} = \left\{\mat{1&1},\mat{-1,1}\right\}$, then 
          $[T]_{\mathcal{B}} = \mat{1&0\\0&0}$.
        \item $T$ is neither one-to-one nor onto. It is not 
          one-to-one because $\vec 0$ and $\mat{-1,1}$ are both sent 
          to the same point. It is not onto because $\mat{0,1}$ is not 
          in the image.
        \item The eigenvalues of $T$ are $1$ and $0$, 
          and $T$ is diagonalizable.
        \item Since $T$ is not one to one, it is not invertible.
      \end{enumerate}
    \item
      \begin{enumerate}
        \item
          \begin{align*}
            -x_1 + x_3 &= 1\\
            x_1 &= 0
          \end{align*}
        \item TODO
        \item TODO
        \item TODO
      \end{enumerate}
  \end{enumerate}
\end{solutions}

\begin{instructions}
  \subsection*{Session Objectives}
  We are pretty open about the fact that this session is for students to find
  their weak spots. We are happy if every student at least attempts every
  problem.

  \subsection*{What to Do}
  Unlike tutorials during the year, in this tutorial, it is okay if students do
  not finish a question before moving on to the next one. Walk around the room
  helping students for a fixed time for each question. Encourage multiple-group
  discussions, and allow groups to exchange ideas, like a mini-tutorial. After
  the allotted time for a question, there will be a short wrap-up. Rinse and
  repeat.

  Some students will want to skip ahead to do things they either know how to
  do, or think they should practice. Do not to let students do this; instead
  encourage them to discuss with other groups.
  \subsection*{Notes}
  \begin{itemize}
    \item
      Question \ref{q:mistake} has math notation mistakes and math content
      mistakes. Try to get each group to notice at least one of each.
    \item
      Students will struggle with question \ref{q:lin_transform}, but it is
      important that they are able to do \ref{q:definition_expansion} and
      \ref{q:lin_almost_lin}.
    \item
      Question \ref{q:inverses} can be done by inspection, but students should
      do it by writing $A$ as a product of elementary matrices
    \item
      Question \ref{q:determinants} can be done by inspection, but if students
      insist on it being done this way, they should have a very good
      explanation to go along.
    \item
      In question \ref{q:standard_basis}, students might try to do some
      complicated calculus to find the minimal distance. This will work, but
      the shorter approach of using plane geometry is preferred.
  \end{itemize}
\end{instructions}
\end{document}
